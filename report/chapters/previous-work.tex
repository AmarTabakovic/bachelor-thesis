\chapter{Previous Work}
This chapter aims to present ideas from already existing approaches.
First, the author's 
predecessor project on basic terrain LOD rendering is summarized.
Afterwards, an overview of literature and software systems for terrain streaming/paging is given,
in which their central ideas are outlined.

\section{Review of Project 2 -- 3D Terrain with Level of Detail}
This thesis is the logical continuation of 
the author's preceeding project \textit{3D Terrain with Level of Detail} \cite{p2}
as part of the course ``Project 2'' at the Bern University of Applied Sciences.
In said project, the basics of terrain rendering and existing approaches were 
studied and evaluated and a simple terrain renderer named \textit{ATLOD} was developed.
A treatment of terrain rendering with streaming/paging was explicitly not part of the preceeding project.
This section aims to summarize the most important parts of the preceeding project.

\subsection{Implemented Terrain LOD Algorithm}
The implemented LOD algorithm is mainly based on GeoMipMapping 
and certain elements from GPU-based Geometry Clipmaps TODO cite. 
The terrain gets split up into logical blocks of size $blockSize = 2^n + 1$ for some $n \in \mathbb{N}$,
where each block contains information such as current LOD level, its world-space center point, the extent points of its AABB,
current border permutation, and more.
A single grid-like flat mesh of side length $blockSize$ is stored on the vertex and index buffer of the terrain.
The indices are stored such that the flat mesh is split into its center and border area. The first part of the index buffer 
contains indices of the center area at every LOD resolution, from highest to lowest, 
and the second part of the index buffer contains the indices of the border area 
at every LOD resolution and for every of the $2^4=16$ possible border permutations. 
A border permutation is a 4-tuple $(l,r,t,b)$ (the variables corresponding to left, right, top and bottom) 
representing a border area, set up such that each element of the 4-tuple is set to 1 if the neighboring block 
on the corresponding side has a lower LOD level, otherwise 0. The arrangement 
of indices with border permutations is for preventing cracks from occurring in the terrain.
The heightmap is stored as a texture image on the GPU.

At runtime, in a first step the LOD level and current border permutation of each block 
is updated. The new LOD level is calculated using the 3-dimensional distance between the camera 
and the block's center point.
In a second step, each block is intersected with the view-frustum 
and if they intersect, the block gets rendered with two draw-calls.
In the vertex shader, the flat mesh gets translated by the block's world-space center point
and the heightmap is sampled for a $y$-value, which is then used to displace the $y$-coordinate of the vertex. 
In the fragment shader, the overlay texture is applied and Phong shading is calculated.
The normal vectors for Phong shading are calculated using the four orthogonally neighboring points 
in the heightmap. Finally, a simple distance fog is applied.

\subsection{Strengths and Limitations}
The main strengths of the implemented algorithm are
its low GPU memory usage.

On the other hand, the main limitations are that it lacks vertex morphing for preventing
pops during LOD level changes, the organization of blocks into a quadtree for 
more efficient view-frustum culling and some other optimizations 
such as instanced rendering.

\section{Literature}
Real-time terrain data streaming has been the topic of research, albeit with a lower 
focus in comparison to terrain LOD algorithms \cite{vtpothers}.

\subsection{Level of Detail for 3D Graphics}
Subchapter ``7.2.5 Paging, Streaming, and Out of Core''
in the the book \textit{Level of Detail for 3D Graphics} 
by Luebke et al.~describes some basic background, existing approaches and relevant publications 
related to terrain streaming. Although the book was published in the year 2003 and can be considered somewhat outdated 
today, certain basic ideas might still hold today.

\subsection{Geometry Clipmaps}
GPU-based Geometry Clipmaps by Asirvatham and Hoppe is a terrain LOD algorithm
based on nested grid-like rings centered around the camera. 

\subsection{Adaptive Streaming and Rendering of Large Terrains:
A Generic Solution}
https://dspace5.zcu.cz/bitstream/11025/10885/1/Lerbour.pdf

\subsection{CDLOD}
Strugar describes some basic information on how the CDLOD algorithm can be extended to
work with streaming \cite{cdlod}.
Additionally, his public implementation of the CDLOD algorithm contains 
a streaming-based variant (\textit{StreamingCDLOD}), which dynamically loads and offloads 
terrain tiles based on the user's movement TODO cite.

\section{Video Games and Game Engines}
\subsection{Unreal Engine}
TODO check out documentation 
https://docs.unrealengine.com/5.3/en-US/level-streaming-in-unreal-engine

\subsection{Far Cry 5}
TODO check out 
https://www.gdcvault.com/play/1025480/Terrain-Rendering-in-Far-Cry

\subsection{Microsoft Flight Simulator}
\textit{Microsoft Flight Simulator} is a video game developed by Asobo Studio and released in 2020.
The video game allows the player to fly on the entire earth and uses various 
data sources for representing the earth as accurately as possible.
It received high praise for its TODO.

At the GDC 2022, Fuentes presented the terrain system of Microsoft Flight Simulator,
in which he described the data organization, system architecture, rendering process and many more aspects.

The terrain system uses a number of sources for height data, overlay texturing, vegetation, bodies of water, cities
and more.

TODO check out
https://www.gdcvault.com/play/1027581/Advanced-Graphics-Summit-Designing-the

\subsection{Battlefield 3}
TODO check out
https://media.contentapi.ea.com/content/dam/eacom/frostbite/files/gdc12-terrain-in-battlefield3.pdf

\subsection{Red Engine 3}
TODO Check out https://www.gdcvault.com/play/1020197/Landscape-Creation-and-Rendering-in

\section{Geographic Information Systems}

\subsection{OpenWebGlobe}
\textit{OpenWebGlobe} is a project and software development kit written in JavaScript and 
WebGL for rendering 3D globes on web browsers. 
It was developed in TODO by TODO. 

\subsection{Google Earth}
\textit{Google Earth} is a 3D globe viewer by Google which is capable of rendering the earth, utilizing a number of data sources for its imagery and height data,
which include Sentinel, TODO.
