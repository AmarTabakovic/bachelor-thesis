\chapter{Conclusion}

\section{Potential Improvements}
This section lists some potential technical improvements
to the rendering and the streaming aspects of StreamingATLOD.

\paragraph{Higher Precision Support}
Currently, StreamingATLOD operates on 32-bit floating point numbers 
for most purposes, including vertex transformations. 
So far, this has not been a problem, since 
the terrain data was limited to zoom level 14, which corresponds 
to a maximum precision of 9.5 meters per pixel. 
However, if StreamingATLOD were to be used with centimeter-accurate terrain data,
this would likely result in jittering artifacts 
and depth buffer fighting during rendering, requiring 
modifications described in section in section ``Potential Problems during Rendering'' of chapter ``Theoretical Background''.

\paragraph{Request Priorization}
When new tiles must be loaded into memory, StreamingATLOD pushes 
these requests into a request queue. Currently,
a worker thread processes these requests in the order 
which they came in, meaning from oldest to newest,
which is not ideal, since old requests which might not 
be needed anymore still have to get loaded.
An improvement would be to prioritize requests.

\paragraph{Transitions Between LOD Changes}
Right now, when a terrain node with zoom level 
gets rendered and at some point 
its four children must get rendered instead,
the sudden change between the lower resolution 
and the higher resolution nodes causes 
a visible popping artifact of higher detailed terrain 
to occur. One way to reduce popping 
is to quickly blend the terrain meshes
together, thus reducing the popping. 
This comes at the cost of having to render 
both the high resolution and low resolution 
meshes at the same time for a short while.

\section{Possibilities for Extension}
This section describes some potential 
features and ideas on how StreamingATLOD could be further enhanced.

\paragraph{Extension into a Software Development Kit}
StreamingATLOD could be extended into a software development kit (SDK)
for developing desktop-based Earth renderers with C++ and OpenGL.
For this, various classes would have to be restructured 
for more flexibility and configurability.

\paragraph{Multiple Data Layers}
Currently, StreamingATLOD operates on one height data layer 
and one imagery layer. StreamingATLOD could be extended 
to be able to work with multiple data layers. This would be 
especially useful in case data is not available in one layer 
but might be available in another.

\paragraph{Realistic Atmospheric Rendering}
StreamingATLOD currently only renders the black color if the camera 
is in space and a static skybox for when the camera is close 
to the ground and interpolates between both if the camera 
is in between. A realistic atmospheric model, such as 
\textit{Precomputed Atmospheric Scattering} by Bruneton and Neyret \cite{precomputedatmosphericscattering}
would increase the realism and the visual aesthetics.

\section{Personal Conclusion}
This conclusion is my personal reflexion and
outlook on this bachelor thesis. Generally, I am very 
satisfied with the system I've developed. This was probably the most 
challenging project I've worked on. 
I am very happy with the results
and was able to finish almost everything I set out to do.