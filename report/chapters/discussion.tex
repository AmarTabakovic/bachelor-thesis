\chapter{Discussion}
This chapter discusses the results obtained in the previous chapter.

\section{Rendering Performance}
In terms of rendering performance, StreamingATLOD performs decently well
given the hardware it was tested on.
With the used configuration settings, the triangle count is kept well below 200000 and the number of draw calls below 
300 even at the highest zoom levels, without 
sacrificing the visual appearance too much.
The framerates are approriate as well in most cases, with only a few occasional hiccups 
while close to the ground.

\section{Streaming Performance}
The average time between arriving at a place and finishing loading the necessary tiles were on average 
18.2 seconds. Given the fact that the disk cache was empty in all experiments, some latency was to be expected.
As such, the main bottleneck of the streaming performance lies in the requesting from the web APIs.
Disk caching prevents this temporarily, especially
for places which are visited frequently by the user.

\section{Memory Usage}
In the experiments, the total memory usage ranged between 800 MB to 1.2 GB, 
which is a normal range for a large-scale Earth renderer.
Upon reaching the memory cache capacity size, the 
memory allocated for terrain data remains practically constant,
since we always evict unused terrain nodes upon loading new ones in.
Once the disk cache is full, then the memory consumption in total stays constant,
since we do not have to store new tile keys in the in-memory disk cache anymore.

\section{Disk Cache Measurements}
The file system size of the disk cache is kept mostly in a normal range,
which is between 92 MB for a disk cache capacity of 400 nodes 
and 1.4 GB for a disk cache capacity of 8000 nodes.
The heightmap subfolder of the file system disk cache 
is in most cases around three to four times 
larger than the overlay subfolder. The reason 
for this is that the heightmap tiles are stored 
in WebP format instead of JPG.

