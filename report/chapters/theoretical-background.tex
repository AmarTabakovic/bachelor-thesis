\chapter{Theoretical Background}
This chapter describes some background information 
on computer graphics, terrain LOD and network programming.

\section{Notation and Terminology}
\subsection{Mathematical Notation}

\subsection{Terminology}

\section{Terrain Rendering}
Terrain rendering refers to the process of displaying the terrain on the screen.

\subsection{Data Representation}
Terrain data can be represented in a number of ways.

\subsection{Level of Detail}
\textit{Level of Detail (LOD)} is the concept of reducing the complexity of a mesh 
using various metrics to optimize rendering performance.
Such metrics include the distance of the mesh to the camera,
the dimensions of the mesh in screen-space and more.

Terrain LOD algorithms can be categorized as follows.

\paragraph{}

Some of the most important terrain LOD algorithms are summarized in 
chapter TODO.

\section{Terrain Streaming}
For visualizing very large terrains, 
the main memory is often not large enough to hold the entire terrain data.
In this case, terrain rendering systems often deploy 
\textit{terrain streaming} (also called \textit{terrain paging}), which refers to the concept of dynamically 
loading and offloading terrain data from the disk or from a network, such as the internet.

