\chapter{Introduction}
Terrains are an important part of many practical applications of 3D computer graphics.
They can be found in video games, simulation software and geographical information systems (GIS).
Rendering terrains, however, is far from an easy task
and there are numerous issues which need to be addressed.

The first main issue is that terrains are expensive to render 
due to their near-constant visibility 
and their sheer size. For example, 
consider a square terrain which consists of 
$8192 \times 8192$ vertices.
Rendering the entire terrain without any optimizations would require 
more than 134 million triangles\footnote{Each vertex represents one height point and each quad consisting of 4 vertices consists of two triangles. The total number of triangles is given by $(8192 - 1) \times (8192 - 1) \times 2 = 134184962$.} 
to be rendered per frame, which is completely infeasible 
even with the most performant GPUs of today. 
The solution for this issue is \textit{level of detail (LOD)}: 
sections of the terrain which are far away from the camera or with low variation in height 
are rendered with fewer vertices and triangles.
Terrain LOD has been the topic of considerable research throughout 
the last three decades, spawning numerous 
algorithms and approaches which solve this issue 
in a variety of creative ways. 

The second main issue concerns the management of terrain data.
Terrain datasets can become extremely large, especially 
when representing real places, such as the Earth.
For example, in 2016 the Google Earth database was estimated to 
contain more than 3000 terabytes of data in total \cite{googleearthdatabase}.
Most consumer computers today have 
between 8 to 32 gigabytes of RAM and 2 to 6 gigabytes of GPU memory,
making loading the entire dataset into memory impossible.
The solution for this issue is \textit{streaming}: instead of 
loading the entire terrain data all at once into memory,
the idea is to only \textit{stream in}
the currently needed data from the disk or over the network.
Over time, unused terrain data must be deallocated
as well in order to make space for new data.

When terrain streaming is performed over the network, 
a problem that can arise is that the 
the terrain data servers get overwhelmed with lots of requests.
The solution for this problem is \textit{caching}:
the concept of temporarily keeping recently or frequently used 
terrain data on the client in order to reduce 
the number of network requests. Terrain caching is usually 
performed in multiple layers. Terrain data can 
be situated (in the order of accessing speed):
on the GPU, in memory, on the disk or on the terrain server \cite[p.~382]{3denginedesignforvirtualglobes}.
Both terrain streaming and caching draw inspiration mainly from 
the areas of operating systems, web browsers and data management.

\section{Goals of this Thesis}
The main goal of this thesis is the development of a prototype for 
a streaming-based terrain rendering system 
with level of detail.
The system should meet the following requirements.

\paragraph{Terrain Rendering} The system should be capable of rendering 
                               a terrain at interactive framerates using 
                               level of detail and other optimizations to ensure appropriate rendering 
                               performance. Preferably, the terrain should be rendered
                               as a sphere for better realism.
\paragraph{Terrain Streaming}  The system should be capable of loading 
                               and offloading terrain data in and out of memory based on 
                              the camera's state.
                               Preferably, the terrain data can be loaded at runtime
                              from web-based APIs served by data providers.
\paragraph{Terrain Caching} The system should be capable of caching terrain 
                             data in order to avoid unnecessary requests
                             for recently or frequently used data.
\paragraph{Additional Features} Some additional, but not crucial, features include proper camera handling,
                               collision detection, atmospheric rendering 
                               and support for a configuration file for 
                               adjusting various parameters.

\subsection{Restrictions}
The following aspects were explicitly left out of scope of this thesis.

\paragraph{Rigorous Testing and Deployment} The developed system serves as 
                             a first proof-of-concept since this thesis 
                              focuses mainly on the development of the features.
                              For future further development after this thesis, rigorous testing and deployment
                              is indespensable.

\paragraph{Development and Deployment of Custom Terrain Data Servers} This thesis does not 
                                                         include the development and deployment of custom terrain data servers.
                                                         The focus of this thesis is on the client-side rather than on the server-side.

\paragraph{Vector Tiles, 3D Buildings and Other Geospatial Features} This thesis focuses mainly on terrains.
                                                                Incorporating other geospatial features, such as vector tiles,
                                                                3D buildings, points-of-interests, etc. would 
                                                                not be realistic in such a short timespan.

\section{Intended Readership}
The reader is assumed to be familiar with the basics of computer science,
C++ programming and 3D computer graphics. The main concepts of terrain rendering will be introduced 
when required in the subsequent chapters.

\section{Notation and Terminology}
This thesis uses the following mathematical notation:
\begin{itemize}
  \item By default, the coordinate system is a right-handed coordinate system with $y$ as the
  up-direction, unless explicitly stated otherwise.
  \item The set of natural numbers is denoted $\mathbb{N}$ and the set of real numbers is denoted $\mathbb{R}$.
  \item Points in $\mathbb{R}^3$ are denoted $\mathbf{p} = (p_x, p_y, p_z)$.
  \item Vectors in $\mathbb{R}^3$ are denoted $\mathbf{v} = (v_x, v_y, v_z)$.
  \item Matrices in $\mathbb{R}^{n \times n}$ are denoted $\mathbf{M}$ and are column-major.
  \item XYZ tile keys (introduced in chapter ``Theoretical Background'') are denoted $tk = (tk_x, tk_y, tk_z)$.
  \item Geodetic coordinates (introduced in chapter ``Theoretical Background'') are denoted $\mathbf{p}_{geodetic} = (lon,lat,h)$.
  \item The vector containing the globe radii for the geodetic-to-cartesian transformation (introduced in chapter ``Theoretical Background'') is denoted $\mathbf{r}_{ellipsoid} = (r_x, r_y, r_z)$.
\end{itemize}

\section{Structure of this Thesis}
This thesis is structured as follows:
\begin{itemize}
  \item Chapter 2 introduces the reader to various topics covered in this thesis, such as 
  the basics of terrain rendering and some concepts from geographical information systems.
  \item Chapter 3 provides a short overview of previous work conducted in the area of real-time terrain rendering and streaming and 
        gives a short recapitulation of the author's preceeding project ``3D Terrain with Level of Detail''.
  \item Chapter 4 goes into the implementation details of \textit{StreamingATLOD}, the implemented streaming-based terrain renderer.
        First, some preliminary information is given, such as the used data and technologies.
        Afterwards, various aspects of StreamingATLOD are described in multiple sections.
  \item Chapter 5 describes the results, including a selection of screenshots of the Earth and various performance benchmarks.
  \item Chapter 6 gives a discussion of the obtained results.
  \item Chapter 7 concludes this thesis with a summary and some potential improvements and future work.
  \item Appendix A contains the time schedule for this thesis.
  %\item Appendix B contains the usage guide for StreamingATLOD with installation and building steps.
\end{itemize}
